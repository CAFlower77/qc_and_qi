\chapter{群论}
\section{基本定义}
\begin{definition}[群]
    (1) 封闭性 (2) 结合律 (3) 单位元 (4) 逆元
\end{definition}
\begin{definition}[有限群]
    若群$G$有限, 则其成员个数$|G|$称为阶数。
\end{definition}
\begin{definition}[Abel群]
    运算\colorbf{可交换}的群, 如整数模$n$的加法群$\mathbb{Z}_n$。
\end{definition}
\begin{definition}[阶数]
    若$g\in G$, 使得$g^r=e$的最小正整数$r\in\mathbb{Z}_{>0}$称为其阶数。
\end{definition}
\begin{definition}[子群]
    $H\leq G$是指$H\subset G$且$H$在$G$运算下构成群。容易看出单位元$e\in G$。
\end{definition}
\begin{exercise}[教材B.1]
    证明有限群的成员都有阶数, 即$\forall g\in\text{有限群}, \exists r\in\mathbb{Z}_{>0},\st g^r=e$。
\end{exercise}
\begin{proof}
    若某个成员$g$没有阶数, 则群$G$有无限大的子集$\qty{g^r:r\in\mathbb{Z}_{>0}}$, 矛盾。进一步, 我们知道群$G$的任何成员的阶数不超过$|G|$。
\end{proof}
\begin{exercise}[教材B.2, Lagrange定理]
    若$H\leq\text{有限群$G$}$, 则$|H|$可整除$|G|$, 除数$[G:H]=\frac{|G|}{|H|}$称为子群$H$的\sffamily{Lagrange指数}。
\end{exercise}
\par 为了证明此定理我们需要引入一些概念:
\begin{definition}[陪集]
    设$H\leq g$, 集合$gH=\qty{gh:h\in H}$, $Hg=\qty{hg:h\in H}$称为$g$对$H$的\colorbf{左陪集}和\colorbf{右陪集}。
\end{definition}
\begin{proposition}
    $gH=H\Longleftrightarrow g\in H$
\end{proposition}
\begin{proof}
    $g\in H\Longrightarrow gH=H$是显然的, 反过来时注意到$e\in H$, 则$g=ge\in gH=H$。
\end{proof}
\begin{proposition}
    $g_1H\cap g_2H=\begin{dcases}
        \text{非空集合} & (g_1H=g_2H)\\
        \varnothing & (g_1H\neq g_2H)
    \end{dcases}$
\end{proposition}
\begin{proof}
    对$\forall g\in g_1H$有$g=g_1h=g_2\qty(g_2^{-1}g_1h)\ (h\in H)$。若$\exists\text{成员$g$}\in g_1H\cap g_2H$, 则有$h_1,h_2\in H\st g=g_1h_1=g_2h_2$ $\Longrightarrow$ $g_2^{-1}g_1=h_2h_1^{-1}\in H$, 由此$g=g_2\qty(g_2^{-1}g_1h)\in g_2H$。类似的$\forall g\in g_2H$ $\Longrightarrow$ $g\in g_1H$, 这就证明了$g_1H=g_2 H$。
\end{proof}
\begin{proposition}
    全部陪集的并$\bigcup_{g\in G}gH=G$。
\end{proposition}
\begin{proof}
    \par 由$gH\subset G$可知$\bigcup_{g\in G}gH\subset G$。
    \par 另一方面, 由于$e\in H$, 故$\forall g\in G, g=ge\in gH$, 这说明$\forall g\in G:G\subset gH$ $\Longrightarrow$ $g\subset\bigcup_{g\in G}gH$。
    \par 综合上述所论, $\bigcup_{g\in G}gH\subset G\land g\subset\bigcup_{g\in G}gH$ $\Longrightarrow$ $G=\bigcup_{g\in G}gH$。
\end{proof}
\begin{proof}[教材习题B.2, Lagrange定理]
    我们知道全部陪集$\{gH:g\in G\}$是一组不交的集合, 容易看出$\bigcup$ $\{gH:g\in G\}=G$, 即$G=\bigsqcup\{gH:g\in G\}=\bigsqcup gH$, 这说明$|G|=\sum |gH|$。容易证明$|gH|=|H|$, 这说明$|G|=\sum |gH|=\sum |H|=|H|\sum 1$, 由此命题得证。
\end{proof}

\begin{exercise}[教材B.3]
    证明每个成员$g\in G$的阶数可以整除$|G|$。
\end{exercise}
\begin{proof}
    \par 令$H=\qty{g^r:1\leq r\leq r_k}=\qty{g,g^2,\cdots,g^{r_k-1},g^{r_k}=e}$, 则容易看出$H\leq G$, 根据Lagrange定理, $H$的指数$[G:H]=\frac{|G|}{|H|}\in\mathbb{Z}_{>0}$, 即$|G|=[G:H]\cdot |H|=[G:H] r_k$ $\Longrightarrow$ $r_k\mid |G|$。
\end{proof}

\begin{definition}
    若$\exists g\in G,\st\text{群成员$a,b\in G$满足$b=g^{-1}ag$}$, 则称$a,b$为共轭成员。
\end{definition}
\begin{proposition}
    群成员间的共轭是等价关系。
\end{proposition}
\begin{proof}
    \begin{enumerate}
        \item 任意成员$a$与其自身共轭:
        $$a=e^{-1}ae$$
        \item 若成员$a$与成员$b$共轭, 则成员$b$与成员$a$共轭:
        $$\qty(b=g^{-1}ag)\Longrightarrow\qty[a=\qty(g^{-1})^{-1}b\qty(g^{-1})]$$
        \item 若成员$a$与成员$b$共轭, 成员$b$与成员$c$共轭, 则成员$a$与成员$c$共轭:
        $$\qty(b=g^{-1}ag)\land\qty(c=g'^{-1}bg')\Longrightarrow\qty[c=g'^{-1}gagg'=\qty(gg')^{-1}a\qty(gg)]$$
    \end{enumerate}
\end{proof}
\begin{definition}[正规子群]
    若$H\leq G$且$\forall g\in G: g^{-1}Hg=H$ (或等价的$Hg=gH$), 则称$H$为$G$的\colorbf{正规子群}, 记作$H\unlhd G$, 记号$H\lhd G$表示$H\neq G$的正规子群。
\end{definition}
\par 我们指出, Abel群的任何子群都是正规子群。设$H\leq\text{Abel群$G$}$, 由于子群$H$的成员也是其继承的Abel群的成员, 这些成员与$G$中任意成员$g$也是可交换的, 故$Hg=gH$显然成立。
\par 对群$G$, $x\in G$的共轭类定义为$G_x\equiv\qty{g^{-1}xg:g\in G}$。容易看出$y\in G_x\Longrightarrow x\in G_y$。注意到
\[\begin{split}
    y\in G_x&\Longrightarrow \exists g\in G:y=g^{-1}xg\\
    &\Longrightarrow \exists g\in G: x=gyg^{-1}=\qty(g^{-1})^{-1}y\qty(g^{-1})\\
    &\xRightarrow{g^{-1}\in G} \exists g\in G: x=g^{-1}yg\Longrightarrow x\in G_y
\end{split}\]
\begin{exercise}[教材B.4]
    $y\in G_x\Longrightarrow G_y=G_x$
\end{exercise}
\begin{proof}
    由于$y\in G_x$, $\exists g_0\in G:y=g_0^{-1}xg_0$或$x=g_0yg_0^{-1}$。现在设$t\in G_y$, 即$\exists g\in G:t=g^{-1}yg=g^{-1}g_0^{-1}xg_0g=(g_0g)^{-1}x(g_0g)$ $\Longrightarrow$ $t\in G_x$, 这就证明了$G_y\subset G_x$。反过来设$t\in G_x$, 即$\exists g\in G:t=g^{-1}xg=g^{-1}g_0yg_0^{-1}g=\qty(g_0^{-1}g)^{-1}y\qty(g_0^{-1}g)$ $\Longrightarrow$ $t\in G_y$。这就证明了$G_x\subset G_y$。综合上述, 我们有$G_y=G_x$。
\end{proof}
\begin{exercise}[教材B.5]
    $x\in\text{Abel群$G$}\Longrightarrow G_x=\{x\}$
\end{exercise}
\begin{definition}[生成元]
    设$g_1,g_2,\cdots,g_\ell\in\text{群$G$}$, 则$G$中全部可以写成$g_1,g_2,\cdots,g_\ell$中若干个成员之乘积的群成员构成$G$的一个子集, 叫做$g_1,g_2,\cdots,g_\ell$所生成的子群$H$, 记作$H=\expval{g_1,g_2,\cdots,g_\ell}$, 即
    $$\expval{g_1,g_2,\cdots,g_\ell}=\qty{g_1^{n_1}g_2^{n_2}\cdots g_\ell^{n_\ell}=\prod_{k=1}^\ell g_k^{n_k}:(n_1,n_2,\cdots,n_\ell)\in\qty(\mathbb{Z}_{\geq 0})^\ell}$$
    $g_1,g_2,\cdots,g_\ell$称为子群$\expval{g_1,g_2,\cdots,g_\ell}$的生成元。
\end{definition}
\par 如果基群$G$是很大的或平凡的, 那么由给定成员生成的继承$G$的子群也称为($G$的)一个\colorbf{生成群}, 我们可以略去$G$的表述。容易看出生成一个阶数为$n$的生成群至少需要$\log(n)$个成员。