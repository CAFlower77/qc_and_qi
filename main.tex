\documentclass[
    8pt,
    mathfont=none,
    a4paper,
    class=book
]{easybook}

\title{量子计算与量子信息 Notes}
\author{Flower CA77}
\date{}

\ctexset{
    style={
        color=seaside,
        uppercase=none
    },
    refset/linktype=colors,
    geoset/hmargin=1.5cm,
    hdrset/fnmarktext-num=pifont
}

\usepackage{
    amsmath, amssymb, esint, mathrsfs, bm, bbm, mathtools,
    cases, extarrows, siunitx, physics, geometry, appendix
}

\PassOptionsToPackage{dvipsnames, svgnames, x11names}{xcolor}
\PassOptionsToPackage{tbtags}{amsmath}

\declaretheoremstyle[
    headfont=\sffamily\color{Green},
    bodyfont=\color{SkyBlue!65!Green},
    notefont=\color{Green!40!SkyBlue},
    headpunct=,
    headindent=\parindent,
    postheadspace=\parindent,
    notebraces = {\ (}{)\ }
]{default_style}
\theoremstyle{default_style}
\declaretheoremstyle[
    headfont=\sffamily\color{SlateBlue},
    bodyfont=\color{SkyBlue!65!Black},
    notefont=\color{SlateBlue!40!SkyBlue},
    headpunct=,
    headindent=\parindent,
    postheadspace=\parindent,
    notebraces = {\ (}{)\ }
]{def_style}
\declaretheorem[
    name=定义,
    numbered=unless unique,
    within=chapter,
    style=def_style
]{definition}
\declaretheorem[
    name=定义,
    numbered=no,
    within=chapter,
    style=def_style
]{definition*}
\declaretheoremstyle[
    headfont=\sffamily\color{Purple},
    bodyfont=\color{SkyBlue!65!Purple},
    notefont=\color{Purple!40!SkyBlue},
    headpunct=,
    headindent=\parindent,
    postheadspace=\parindent,
    notebraces = {\ (}{)\ }
]{thm_style}
\declaretheorem[
    name=定理,
    numbered=unless unique,
    within=section,
    style=thm_style
]{theorem}
\declaretheorem[
    name=命题,
    numbered=unless unique,
    style=thm_style
]{proposition}
\declaretheoremstyle[
    headfont=\sffamily\color{DarkCyan},
    bodyfont=\ttfamily\color{SkyBlue!75!Black},
    notefont=\color{DarkCyan!45!SkyBlue},
    headpunct=,
    headindent=\parindent,
    postheadspace=\parindent,
    notebraces = {\ (}{)\ }
]{ex_style}
\declaretheorem[
    name=例,
    numbered=unless unique,
    style=ex_style
]{example}
\declaretheorem[
    name=习题,
    numbered=no,
    style=ex_style
]{exercise}
\declaretheoremstyle[
    headfont=\sffamily\color{Red!70!Black},
    bodyfont=\ttfamily\color{SkyBlue!65!Red},
    notefont=\color{Red!40!SkyBlue},
    headpunct=,
    headindent=\parindent,
    postheadspace=\parindent,
    notebraces = {\ (}{)\ }
]{sol_style}
\declaretheorem[
    name=解,
    numbered=no,
    style=sol_style
]{solution}
\declaretheorem[
    name=证明,
    numbered=no,
    style=sol_style
]{proof}

\newcommand{\colorbf}[1]{\ \textcolor{DarkCyan}{\textbf{#1}}\ }
\DeclareMathOperator{\Span}{{\rm Span}}
\DeclareMathOperator{\st}{{\rm \ s.t.\ }}
\newcommand{\E}{\mathbb{E}}
\DeclareMathOperator{\Var}{{\rm var}}
\renewcommand{\leq}{\leqslant}
\renewcommand{\geq}{\geqslant}
\newcommand{\Io}[1]{\mathop{\left]#1\right[}}
\newcommand{\Ic}[1]{\mathop{\left[#1\right]}}
\newcommand{\Ioc}[1]{\mathop{\left]#1\right]}}
\newcommand{\Ico}[1]{\mathop{\left[#1\right[}}
\renewcommand{\land}{\mathop{\texttt{\&\&}}}
\renewcommand{\lor}{\mathop{\texttt{||}}}
\renewcommand{\lnot}{\mathop{\texttt{!}}}

\lineskip=3pt
\lineskiplimit=3pt
\parskip=5pt

\begin{document}

\maketitle
\tableofcontents

\part{基础概念}

\chapter{简介与概述}
\section{量子比特}
\section{量子计算}
\section{量子算法}
\section{实验量子信息处理}
\section{量子信息}

\chapter{量子力学引论}
\section{线性代数}
\section{量子力学的假设}
\section{应用: 超密编码}
\section{密度算子}
\section{施密特分解与纯化}
\section{EPR和Bell不等式}

\chapter{计算机科学简介}
\section{计算模型}
\section{计算问题的分析}
\section{关于计算科学的观点}

\part{量子计算}

\chapter{量子电路}
\section{量子算法}
\section{单量子比特操作}
\section{受控操作}
\section{量子傅里叶变换的一般应用}
\section{通用量子门}
\section{量子计算电路模型总结}
\section{量子系统的模拟}

\chapter{量子Fourier变换及其应用}
\section{量子Fourier变换}
\section{相位估计}
\section{应用: 求阶与因子分解问题}
\section{量子傅里叶变换的一般应用}

\chapter{量子搜索算法}
\section{量子搜索算法}
\section{作为量子模拟的量子搜索}
\section{量子计数}
\section{NP完全问题解的加速}
\section{无结构数据库的量子搜索}
\section{搜索算法的最优性}
\section{黑盒算法的极限}

\chapter{量子计算机: 物理实现}
\section{指导性原则}
\section{量子计算的条件}
\section{谐振子量子计算机}
\section{光学光量子计算机}
\section{光学腔量子电动力学}
\section{离子阱}
\section{核磁共振}
\section{其他实现方案}

\part{量子信息}

\chapter{量子噪声与量子操作}
\section{经典噪声与Markov过程}
\section{量子操作}
\section{量子噪声与量子操作的例子}
\section{量子操作的应用}
\section{量子操作形式体系的局限}

\chapter{量子信息的距离度量}
\section{经典信息的距离度量}
\section{两个量子态有多接近}
\section{量子信道保护信息的效果怎么样}

\chapter{量子纠错}
\section{背景介绍}
\section{熵的基本性质}
\section{量子纠错理论}
\section{构造量子编码}
\section{稳定子编码}
\section{容错量子计算}

\chapter{熵与信息}
\section{Shannon熵}
\section{熵的基本性质}
\section{冯·诺伊曼熵}
\section{强次可加性}

\chapter{量子信息论}
\section{量子态的区分与可达信息}
\section{数据压缩}
\section{噪声信道上的经典信息}
\section{有噪声量子信道的量子信息}
\section{作为一种物理资源的纠缠}
\section{量子密码学}

\addcontentsline{toc}{part}{附录}
\part*{附录}

\appendix

\chapter{概率论基础}
\par 随机变量$X$取值$x$的概率为$p(X=x)$:
\begin{itemize}
    \item 如果$X$是离散型随机变量, 则$p(X=x)$为$X=x$的概率
    \item 如果$X$是连续型随机变量, 则$p(X=x)=0$, 因为一个点在一个连续区域内的测度为零, 此时我们引入概率密度$\rho(x)$, 使得$p(x\leq X<x+dx)=\rho(x)dx$
\end{itemize}
我们简记$p(X=x)$为$p_X(x)$, 不引起混淆时进一步简记为$p(x)$。
\par 常数$c$可以看成取值$X=c$概率为$1$的随机变量$X$。
\par 概率密度的概念是始终有效的, 对离散型随机变量$X$, 我们可以取$\rho(x)=\sum_{x'}p(X=x')\delta(x-x')$, 其中求和$x'$取遍$X$的所有值, $\delta$为Dirac的$\delta$函数。进一步, 当$x'$不是$X$可取的值时$p(X=x')=0$, 求和$x'$可以取遍全部值。
\par 一组随机变量$X_1,X_2,\cdots,X_n$组成随机向量$\bm X=(X_1,X_2,\cdots,X_n)$, 联合分布$p(\bm X=\bm x)=p(X_1=x_1,X_2=x_2,\cdots,X_n=x_n)$也简记为$p_{\bm X}(\bm x)$, 不引起混淆时简单记作$p(\bm x)$。
\par 条件概率$p(Y=y\mid X=x)=p(X=x,Y=y)/p(X=x)$, 简记为$p_{Y\mid X}(y\mid x)=p_{(X,Y)}(x,y)/p_X(x)$, 其中分子是随机向量$(X,Y)$的联合分布。不引起混淆时我们简单记作$p(y\mid x)$。
\par\colorbf{\textbf{注意}}\quad 在略去概率$p$的角标时[即简记$p_X(x)$为$p(x)$]必须要规范标记变量, 即用单个大写字母$X$表示随机变量, 其小写值$x$表示对应的$X$的取值, 这个规定也是$p(x)$的缺省标准。在表达式比较复杂时, 可以显式写出随机变量$p(X=x)$。
\begin{exercise}[教材A.1] 证明Bayes定律$p(x\mid y)=p(y\mid x)\frac{p(x)}{p(y)}$
\end{exercise}
\begin{proof}
    把待证等式改写为$p(x\mid y)p(y)=p(y\mid x)p(x)$, 可以看到等式两边都是联合分布概率$p(x,y)$, 这就证明了待证方程。
\end{proof}
\begin{exercise}[教材A.2]
    全概率公式 $p(y)=\sum_x p(y\mid x)p(x)$
\end{exercise}
\begin{proof}
    $\sum_x p(y\mid x)p(x)=\sum_x p(x,y)=p(y)$
\end{proof}
\par 期望$\E X\equiv\sum_x xp(x)$, 方差$\Var X\equiv\E\qty[\qty(X-EX)^2]=\E\qty(X^2)-(\E X)^2$, 标准差$\Delta X\equiv\sqrt{\Var X}$
\begin{exercise}[教材A.3]
    证明$\exists x\geq\E X, \st p(x)>0$
\end{exercise}
\begin{proof}
    反证, 只要证明命题$\forall x\geq\E X:p(x)=0$是伪命题即可。考虑到
    $$\E X=\sum_x xp(x)=\sum_{x<\E X}xp(x)+\sum_{x\geq \E X}xp(x)=\sum_{x<\E X}xp(x)<\E X\sum_{x<\E X}p(x)\leq\E X\sum_{x}p(x)=\E X$$
\end{proof}
\begin{exercise}[教材A.4]
    证明$\E X$对$X$是线性的。
\end{exercise}
\begin{proof}
    $\E(kX)=\sum_x kxp(x)=k\sum_x xp(x)=k\E X$
\end{proof}
\begin{exercise}[教材A.5]
    证明$X,Y$独立时$\E(XY)=\E X\cdot \E Y$
\end{exercise}
\begin{proof}$\E(XY)=\sum_{x,y}xyp(x,y)\xlongequal{\text{$X,Y$独立}}\sum_{x,y}xyp(x)p(y)=\sum_x xp(x)\sum_y yp(y)=\E X\cdot \E Y$
\end{proof}
\begin{exercise}[教材A.6, Cheybshev不等式]
    $\forall\lambda>0$和有限方差的$X$, $p\qty(\qty|x-\E X|\geq\lambda\Delta X)\leq\frac{1}{\lambda^2}$
\end{exercise}
\begin{proof}
    我们设概率密度为$\rho(x)$, 则
    \[\begin{split}
        \qty(\Delta X)^2=\Var X&=\E\qty[\qty(X-\E X)^2]=\int(x-\E X)^2\rho(x)dx\\
        &=\int_{x-\E X\leq -\lambda\Delta X}(x-\E X)^2\rho(x)dx+\int_{\E X-\lambda\Delta X}^{\E X+\lambda\Delta X}(x-\E X)^2\rho(x)dx+\int_{x-\E X\geq\lambda\Delta X}(x-\E X)^2\rho(x)dx\\
        &\geq\lambda^2\Delta X^2\int_{x-\E X\leq-\lambda\Delta X}\rho(x)dx+0+\lambda^2\Delta X^2\int_{x-\E X\geq\lambda\Delta X}\rho(x)dx=\lambda^2\Delta X^2\int_{\qty|x-\E X|\geq\lambda\Delta X}\rho(x)dx
    \end{split}\]
    则$p\qty(\qty|X-\E X|\geq\lambda\Delta X)=\int_{\qty|x-\E X|\geq\lambda\Delta X}\rho(x)dx\leq\frac{\Delta X^2}{\lambda^2\Delta X^2}=\frac{1}{\lambda^2}$
\end{proof}
\par 我们补充一些内容, 首先是关于方差的一些定义。
\begin{definition}[协方差]
    两个随机变量$X,Y$的协方差定义为
    $$\Cov(X,Y)=\E\qty[\qty(X-\E X)\qty(Y-\E Y)]=\E(XY)-\E X\cdot \E Y$$
    容易看出同一个随机变量和其自身的协方差即为其自身的方差$\Cov(X,X)=\Var X=\qty(\Delta X)^2$, 两个变量的协方差是对称的$\Cov(X,Y)=\Cov(Y,X)$。
\end{definition}
\begin{definition}[相关系数]
    两个随机变量$X,Y$的相关系数为其协方差除以两个变量的标准差
    $$\rho_{X,Y}=\frac{\Cov(X,Y)}{\Delta X\cdot\Delta Y}=\frac{\Cov(X,Y)}{\sqrt{\Var X}\sqrt{\Var Y}}$$
    容易看出同一个随机变量和其自身是完全相关的$\rho_{X,X}=1$, 两个变量的相关是对称的$\rho_{X,Y}=\rho_{Y,X}$。
\end{definition}
\begin{proposition}
    两个随机变量$X,Y$相关系数$\rho_{X,Y}=1$ $\Longleftrightarrow$ 两者与常数$1$线性相关$\exists\lambda,\mu\in\mathbb R:\lambda X+\mu Y=1$, 此时我们称两个变量\colorbf{完全相关}。若$\rho_{X,Y}=0$, 我们称$X,Y$完全不相关。根据相关系数的符号$\rho_{X,Y}>0$或$<0$时我们称$X,Y$\colorbf{正相关}或\colorbf{负相关}。
\end{proposition}
\begin{definition}
    对随机向量$\bm X=(X_1,X_2,\cdots,X_n)$, 其\colorbf{协方差矩阵}$\bm C=\Cov\bm X=\qty[\Cov(X_i,X_j)]_{n\times n}$和\colorbf{相关系数矩阵}$\bm\rho=\rho_{\bm X}=\qty[\rho_{X_i,X_j}]_{n\times n}$分别为
    $$\mqty[
        \Var X_1 & \Cov(X_1, X_2) & \cdots & \Cov(X_1, X_n)\\
        \Cov(X_2, X_1) & \Var X_2 & \cdots & \Cov(X_2, X_n)\\
        \vdots & \vdots &  & \vdots\\
        \Cov(X_n, X_1) & \Cov(X_n, X_2) & \cdots & \Var X_n
    ],\qquad \mqty[
        1 & \rho_{X_1, X_2} & \cdots & \rho_{X_1, X_n}\\
        \rho_{X_2, X_1} & 1 & \cdots & \rho_{X_2, X_n}\\
        \vdots & \vdots &  & \vdots\\
        \rho_{X_n, X_1} & \rho_{X_n, X_2} & \cdots & 1\\
    ]$$
\end{definition}
\par 然后我们来补充一些和期望有关的工具。
\begin{definition}[条件期望]
    随机变量$X$在取值$x$的情况下随机变量$Y$的条件期望为$Y$在$X=x$下条件概率的期望$\E(Y\mid X=x)=\sum_{y}yp(Y=y\mid X=x)$, 简记为$\E(Y\mid x)=\sum_y yp(y\mid x)$。
\end{definition}
\begin{definition}[生成函数]
    随机变量$X$的生成函数$f_X(t)$或简记为$f(X)$, 定义为$f(X)=\E\qty(e^{itX})$, 将其展开可以发现生成函数事实上是概率的Fourier变换$f(X)=\sum_x p(x)e^{itx}$, 当我们把概率写成概率密度的积分时可以更明显的看出Fourier变换的形式$f(X)=\int \rho(x)e^{itx}dx$。
\end{definition}
\begin{proposition}
    若$X,Y$独立, 则$Z=X+Y$的生成函数为$f(Z)=f(X+Y)=f(X)f(Y)$
\end{proposition}
\begin{proof}
    $f(Z)=\E\qty(e^{itZ})=\E\qty(e^{it(X+Y)})=\E\qty(e^{itX}e^{itY})\xlongequal{\text{$X,Y$独立}}\E\qty(e^{itX})\E\qty(e^{itY})=f(X)f(Y)$
\end{proof}
\par 常数$c$的生成函数为$\E(e^{itc})=e^{itc}$, 是一个转动的相位。
\par 利用生成函数可以证明概率论中最重要的定理:
\begin{theorem}[中心极限定理]
    对均值$\mu$标准差$\sigma$的i.i.d.(独立同分布)随机变量$X_1,X_2,\cdots,X_n$, 当样本数$n$趋向无限大时有极限$\lim_{n\to\infty}p\Biggl(\frac{\sum\limits_{i=1}^nX_i-n\mu}{\sigma\sqrt{n}}\leq x\Biggr)=\frac{1}{\sqrt{2\pi}}\int_{-\infty}^x\exp(-\frac{1}{2}z^2)dz$
\end{theorem}
\begin{proof}
    设$Z_n=\frac{\sum\limits_{i=1}^nX_i-n\mu}{\sigma\sqrt{n}}=\sum_{i=1}^n\frac{X_i-\mu}{\sigma\sqrt{n}}$, 则
    $$f(Z_n)=\prod_{i=1}^n f\qty(\frac{X_i-\mu}{\sigma\sqrt{n}})=\prod_{i=1}^n f\qty(\frac{X_i}{\sigma\sqrt{n}})\exp(-i\frac{\mu t}{\sigma\sqrt{n}})$$
    由于$X_i\ (i=1,2,\cdots,n)$ i.i.d. , 故上式变成
    $$f(Z_n)=\qty[f\qty(\frac{X}{\sigma\sqrt{n}})\exp(-i\frac{\mu t}{\sigma\sqrt{n}})]^n=\qty[\E\exp(i\frac{Xt}{\sigma\sqrt{n}})\exp(-i\frac{\mu t}{\sigma\sqrt{n}})]^n=\qty[\E\exp(i\frac{X-\mu}{\sigma\sqrt{n}}t)]^n$$
    作Taylor展开得到
    $$f(Z_n)=\qty[1+i\E\frac{X-\mu}{\sigma\sqrt{n}}t-\E\frac{(X-\mu)^2}{2\sigma^2n}t^2+O\qty(n^{-3/2})]^n$$
    注意到$n\to\infty$时$\E\overline{X}=\sum\frac{\E X}{n}\to\mu$, 也就是说$n\to\infty$时生成函数里面的全部$\E X$换成$\mu$后得到的量是原来的等价量, 即
    $$f(Z_n)\sim\qty[1-\E\frac{(X-\mu)^2}{2\sigma^2n}t^2]^n\to\exp[-\E\frac{(X-\mu)^2}{2\sigma^2}t^2]$$
    那么
    $$f(Z)=f\qty(\lim_{n\to\infty}Z_n)\sim\lim_{n\to\infty}f(Z_n)=\exp[-\E\frac{(X-\mu)^2}{2\sigma^2}t^2]=\exp(-\frac{1}{2}t^2)$$
    正如我们提到的, 生成函数实际上是概率分布的Fourier变换, 即$\int p(z)e^{itz}dz=\exp(-\frac{1}{2}t^2)$, 那么我们直接将生成函数作Fourier逆变换就能得到概率密度$p(z)=\frac{1}{2\pi}\int\exp(-\frac{1}{2}t^2)e^{-itz}dt=\frac{1}{\sqrt{2\pi}}\exp(z^2)$, 即标准Gauss分布的概率密度。
\end{proof}
\par 这个定理事实上是说, 从均值$\mu$标准差$\sigma$的统计数据中抽样$X$, 当样本容量$n$足够大时样本的均值$\overline{X}=\frac{1}{n}\sum X$将趋向于Gauss分布$N\qty(\mu,\frac{\sigma^2}{n})$, 即$\frac{1}{\sqrt{n}}\overline{X^*}=\frac{\sum X-n\mu}{\sigma\sqrt{n}}$ $\qty(\text{$X^*=\frac{X-\mu}{\sigma}$为$X$的标准化变量})$的概率密度将趋向于标准Gauss分布的概率密度$\frac{1}{\sqrt{2\pi}}\exp(-\frac{1}{2}t^2)$。均值的分布标准差相比样本自身的标准差相差倍数$\frac{1}{\sqrt{n}}$, 这个系数就是所谓的\colorbf{经典测量极限}。
\section{群论}
\subsection{基本定义}
\begin{definition*}[群]
    (1) 封闭性 (2) 结合律 (3) 单位元 (4) 逆元
\end{definition*}
\begin{definition*}[有限群]
    若群$G$有限, 则其成员个数$|G|$称为阶数。
\end{definition*}
\begin{definition*}[Abel群]
    运算\colorbf{可交换}的群, 如整数模$n$的加法群$\mathbb{Z}_n$。
\end{definition*}
\begin{definition*}[阶数]
    若$g\in G$, 使得$g^r=e$的最小正整数$r\in\mathbb{Z}_{>0}$称为其阶数。
\end{definition*}
\begin{definition*}[子群]
    $H\leq G$是指$H\subset G$且$H$在$G$运算下构成群。
\end{definition*}
\begin{exercise}[B.1]
    证明有限群的成员都有阶数, 即$\forall g\in\text{有限群}, \exists r\in\mathbb{Z}_{>0},\st g^r=e$。
\end{exercise}
\begin{proof}
    若某个成员$g$没有阶数, 则群$G$有无限大的子集$\qty{g^r:r\in\mathbb{Z}_{>0}}$, 矛盾。进一步, 我们知道群$G$的任何成员的阶数不超过$|G|$。
\end{proof}
\begin{exercise}[B.2, Lagrange定理]
    若$H\leq\text{有限群$G$}$, 则$|H|$可整除$|G|$, 除数$[G:H]=\frac{|G|}{|H|}$称为子群$H$的\sffamily{Lagrange指数}。
\end{exercise}
\par 为了证明此定理我们需要引入一些概念:
\begin{definition*}[陪集]
    设$H\leq g$, 集合$gH=\qty{gh:h\in H}$, $Hg=\qty{hg:h\in H}$称为$g$对$H$的\colorbf{左陪集}和\colorbf{右陪集}。
\end{definition*}
\begin{proposition}
    $gH=H\Longleftrightarrow g\in H$
\end{proposition}
\begin{proof}
    $g\in H\Longrightarrow gH=H$是显然的, 反过来时注意到$e\in H$, 则$g=ge\in gH=H$。
\end{proof}
\begin{proposition}
    $g_1H\cap g_2H=\begin{dcases}
        \text{非空集合} & (g_1H=g_2H)\\
        \varnothing & (g_1H\neq g_2H)
    \end{dcases}$
\end{proposition}
\begin{proof}
    对$\forall g\in g_1H$有$g=g_1h=g_2\qty(g_2^{-1}g_1h)\ (h\in H)$。若$\exists\text{成员$g$}\in g_1H\cap g_2H$, 则有$h_1,h_2\in H\st g=g_1h_1=g_2h_2$ $\Longrightarrow$ $g_2^{-1}g_1=h_2h_1^{-1}\in H$, 由此$g=g_2\qty(g_2^{-1}g_1h)\in g_2H$。类似的$\forall g\in g_2H$ $\Longrightarrow$ $g\in g_1H$, 这就证明了$g_1H=g_2 H$。
\end{proof}
\begin{proof}[习题B.2, Lagrange定理]
    我们知道全部陪集$\{gH:g\in G\}$是一组不交的集合, 容易看出$\bigcup\{gH:g\in G\}=G$, 即$G=\bigsqcup\{gH:g\in G\}=\bigsqcup gH$, 这说明$|G|=\sum |gH|$。容易证明$|gH|=|H|$, 这说明$|G|=\sum |gH|=\sum |H|=|H|\sum 1$, 由此命题得证。
\end{proof}

\begin{exercise}[B.3]
    证明每个成员$g\in G$的阶数可以整除$|G|$。
\end{exercise}


\begin{definition*}
    若$\exists g\in G,\st\text{群成员$a,b\in G$满足$b=g^{-1}ag$}$, 则称$a,b$为共轭成员。
\end{definition*}
\begin{proposition}
    群成员间的共轭是等价关系。
\end{proposition}
\begin{proof}
    \begin{enumerate}
        \item 任意成员$a$与其自身共轭:
        $$a=e^{-1}ae$$
        \item 若成员$a$与成员$b$共轭, 则成员$b$与成员$a$共轭:
        $$\qty(b=g^{-1}ag)\Longrightarrow\qty[a=\qty(g^{-1})^{-1}b\qty(g^{-1})]$$
        \item 若成员$a$与成员$b$共轭, 成员$b$与成员$c$共轭, 则成员$a$与成员$c$共轭:
        $$\qty(b=g^{-1}ag)\land\qty(c=g'^{-1}bg')\Longrightarrow\qty[c=g'^{-1}gagg'=\qty(gg')^{-1}a\qty(gg)]$$
    \end{enumerate}
\end{proof}
\chapter*{Solovay-Kitaev定理}
\addcontentsline{toc}{chapter}{Solovay-Kitaev定理}
\chapter*{数论}
\addcontentsline{toc}{chapter}{数论}
\chapter*{公钥密码和RSA密码系统}
\addcontentsline{toc}{chapter}{公钥密码和RSA密码系统}
\chapter{Lieb定理的证明}



\end{document}